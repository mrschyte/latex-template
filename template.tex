\documentclass[12pt,a4paper,titlepage]{article}
\usepackage[utf8]{inputenc}
\usepackage[T1]{fontenc}
\usepackage[magyar]{babel}
\usepackage[final]{pdfpages}
\usepackage{graphicx}
\usepackage{amsmath}
\usepackage{amssymb}
\usepackage{amsthm}

\usepackage[bookmarks=true, colorlinks=true, pdfstartview=FitV, linkcolor=blue, citecolor=blue, urlcolor=blue]{hyperref}

%\usepackage{venturis2}
\usepackage{tgtermes}
%\usepackage{palatino}

%\usepackage[scaled]{helvet}
%\usepackage[lf]{venturis}
%\usepackage{arial}
\usepackage{amsfonts}
%\usepackage{lmodern}

%\renewcommand*\familydefault{\sfdefault}

%\usepackage{mathdesign}

\title{Latex template}
\author{Szerző Antal}
\frenchspacing

\begin{document}
\maketitle
\newpage

\tableofcontents
\newpage

% theorems, definitions
\newtheorem{theorem}{Tétel}[section]
\newtheorem{corollary}[theorem]{Következmény}
\newtheorem{lemma}[theorem]{Lemma}

\theoremstyle{remark}
\newtheorem*{remark}{Megjegyzés}
\newtheorem*{assertion}{Állítás}
\newtheorem*{consequence}{Következmény}
\newtheorem*{notation}{Jelölés}

\theoremstyle{definition}
\newtheorem{definition}{Definíció}[section]

\section{Tételek, bizonyítások}
Ebben a részben a latex, tétel és bizonytás környezetét mutatjuk be.

\begin{theorem}[A latex hatékonyságáról]
\label{th:latex:efficiency}
A latex-et könnyű használni.
\end{theorem}

\begin{proof}
Ez gyakorlott szakember számára igaz, de minden latex-et használó ember
gyakorlott szakember, így ebből következik \aref{th:latex:efficiency} állítás.
\[ \sum_{i = 0}^\infty \frac{\binom{n}{k / i}}{2^i} \]
\end{proof}

\begin{corollary}
A {\LaTeX} egy jó program.
\end{corollary}

\begin{definition}[LaTeX]
A latex egy szövegszerkesztő program.
\end{definition}
\end{document}
